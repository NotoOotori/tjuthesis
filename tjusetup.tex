% !TeX root = ./tjuthesis-example.tex

% 论文基本信息配置

\tjusetup{
  %******************************
  % 注意:
  %   1. 配置里面不要出现空行
  %   2. 不需要的配置信息可以删除
  %   3. 建议先阅读文档中所有关于选项的说明
  %******************************
  %
  % 输出格式 (还没有做)
  %   electronic  **    默认
  %   print       **
  %   draft       草稿
  %
  % output = electronic,
  %
  %
  % 样式
  %   standard  按照学校手册要求进行排版 (默认)
  %   math      模仿应数专业下发的"应数毕业论文模板new.tex"的排版
  %
  style = standard,
  %
  %
  % 标题
  %   不可使用“\\”命令手动控制换行
  %   副标题为可选项
  %
  title  = {同济大学学位论文\LaTeX 模板},
  subtitle = {使用示例文档 v\version},
  title* = {An Introduction to \LaTeX\ Thesis Template of Tongji University v\version},
  % subtitle* = {}
  %
}


% Put any packages you would like to use here

% 表格中支持跨行
% \usepackage{multirow}

% 跨页表格
% \usepackage{longtable}

% 固定宽度的表格。放在 hyperref 之前的话,tabularx 里的 footnote 显示不出来。
% \usepackage{tabularx}

% 表格加脚注
% \usepackage{threeparttable}
% \pretocmd{\TPTnoteSettings}{\footnotesize}{}{}

% 确定浮动对象的位置,可以使用 H,强制将浮动对象放到这里(可能效果很差)
% \usepackage{float}

% 浮动图形控制宏包。
% 允许上一个 section 的浮动图形出现在下一个 section 的开始部分
% 该宏包提供处理浮动对象的 \FloatBarrier 命令,使所有未处
% 理的浮动图形立即被处理。这三个宏包仅供参考,未必使用:
% \usepackage[below]{placeins}
% \usepackage{floatflt} % 图文混排用宏包
% \usepackage{rotating} % 图形和表格的控制旋转

% 定理类环境宏包
\usepackage[amsmath,thmmarks,hyperref,thref]{ntheorem}

% One should always load \usepackage{hyperref} before the first use of \newtheorem to obtain correct handling and referencing of counters.
% \AtBeginDocument{
%   \theoremstyle{plain}
%   \theoremheaderfont{\bfseries}\theorembodyfont{\upshape}
%   \theoremindent0pt
%   \newtheorem{theorem}{定理}[subsection]
%   \newtheorem{corollary}[theorem]{推论}
%   \newtheorem{lemma}[theorem]{引理}
%   \newtheorem{proposition}[theorem]{命题}
%   \theoremsymbol{\ensuremath{\mdlgwhtsquare}}
%   \newtheorem{theoremnoproof}[theorem]{定理}
%   \newtheorem{propositionnoproof}[theorem]{命题}
%   \theoremsymbol{}
%   \newtheorem{definition}[theorem]{定义}
%   \newtheorem{remark}[theorem]{注}
%   \newtheorem{example}[theorem]{例}
%   \theoremstyle{nonumberplain}
%   \theoremheaderfont{\bfseries}\theorembodyfont{\upshape}
%   \theoremindent0pt
%   \theoremsymbol{\ensuremath{\mdlgwhtsquare}}
%   \newtheorem{proof}{证明}
%   \theoremsymbol{}
%   % TODO: theorem style
% }

% 给自定义的宏后面自动加空白
% \usepackage{xspace}

% 定义所有的图片文件在 figures 子目录下
% \graphicspath{{figures/}}

% 配置enumitem宏包
% \setlist*[enumerate,1]{left = \parindent .. 2\parindent} % 设置enumerate环境一级分项的缩进, 用star version以添加设置而不是覆盖之前的设置

% 定义自己常用的东西
% \def\myname{陈旭阳}

% 数学命令
% unicode-math does many jobs at begin document, so we need to declare operator after that.
% 算子
% \AtBeginDocument{%https://tex.stackexchange.com/a/117995
%   \let\dim\relax
%   \DeclareMathOperator{\dim}{dim}
%   \DeclareMathOperator{\codim}{codim}
%   %https://tex.stackexchange.com/questions/175251/how-to-redefine-a-command-using-declaremathoperator
% }
% 实用工具宏
% \newcommand*{\lr}[3]{\left#1#3\right#2}
% \newcommand*{\powerset}[1]{2^{#1}}
% 简写
% \newcommand*{\CC}{\symbb{C}}
% \newcommand*{\ZZ}{\symbb{Z}}
% \newcommand*{\NN}{\symbb{N}}

% %%%%%%%%%%%%%%%%%%%%%%%%%%%%%%%%%%%%%%%%%%%%%%%%%%%%%%%%%%%%%%%%%%%%%%%%%%% %
% 从这开始的代码是专门为了示例用的, 使用者可以全部注释掉
% %%%%%%%%%%%%%%%%%%%%%%%%%%%%%%%%%%%%%%%%%%%%%%%%%%%%%%%%%%%%%%%%%%%%%%%%%%% %
% 借用 ltxdoc 里面的几个命令方便写文档。
% \DeclareRobustCommand\cs[1]{\texttt{\char`\\#1}}
% \providecommand\pkg[1]{{\sffamily#1}}

% 借用 lipsum与zhlipsum 来填充内容
\usepackage{lipsum}
\usepackage{zhlipsum}

% 设置提到包与程序名字时候所用的字体
\newcommand{\pckg}[1]{\textsc{#1}}
