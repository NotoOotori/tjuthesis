% !TeX root = ./tjuthesis-example.tex

% 论文基本信息及配置

\tjusetup{
  %******************************
  % 注意:
  %   1. 配置里面不要出现空行
  %   2. 不需要的配置信息可以删除
  %   3. 建议先阅读文档中所有关于选项的说明
  %******************************
  %
  % 输出格式 (还没有做)
  %   electronic  **    默认
  %   print       **
  %   draft       草稿
  %
  % output = electronic,
  %
  % 标题
  %   不可使用“\\”命令手动控制换行
  %   副标题为可选项
  %
  title  = {同济大学学位论文\LaTeX 模板},
  subtitle = {使用示例文档 v\version},
  title* = {An Introduction to \LaTeX\ Thesis Template of Tongji University v\version},
  % subtitle* = {}
  %
  % 培养单位
  %   填写所属学院的全名
  %
  department = {数学科学学院},
  %
  % 学科
  %   本科生填写专业名称
  %
  discipline  = {数学与应用数学},
  %
  % 姓名
  %
  author  = {陈旭阳},
  %
  % 学号
  %
  id = {1753763},
  %
  % 指导教师
  %   中文姓名和职称之间以英文逗号分开, 也可不写职称
  %
  supervisor  = {单海英, 教授},
  % 日期
  %   使用 ISO 格式; 默认为当前时间
  %
  date = {2021-07-11},
  %
  % 英文字体
  %   times   Times New Roman 字体,
  %             如果系统中没有则会使用 TeX Gyre Termes 字体 (默认)
  %   cmu     Computer Modern Serif 字体
  %
  font = times,
  %
  % cover
  %
  cover = true,
  %
}


% Put any packages you would like to use here

% 定义自己常用的东西
% \def\myname{陈旭阳}

% 数学命令
% unicode-math does many jobs at begin document, so we need to declare operator after that.
% 算子
% \AtBeginDocument{%https://tex.stackexchange.com/a/117995
%   \let\dim\relax
%   \DeclareMathOperator{\dim}{dim}
%   \DeclareMathOperator{\codim}{codim}
%   %https://tex.stackexchange.com/questions/175251/how-to-redefine-a-command-using-declaremathoperator
% }
% 实用工具宏
% \newcommand*{\lr}[3]{\left#1#3\right#2}
% \newcommand*{\powerset}[1]{2^{#1}}
% 简写
% \newcommand*{\CC}{\symbb{C}}
% \newcommand*{\ZZ}{\symbb{Z}}
% \newcommand*{\NN}{\symbb{N}}

% 定理类环境宏包
% \usepackage[amsmath,thmmarks,hyperref,thref]{ntheorem}

% One should always load \usepackage{hyperref} before the first use of \newtheorem to obtain correct handling and referencing of counters.
% \AtBeginDocument{
%   \theoremstyle{plain}
%   \theoremheaderfont{\bfseries}\theorembodyfont{\upshape}
%   \theoremindent0pt
%   \newtheorem{theorem}{定理}[subsection]
%   \newtheorem{corollary}[theorem]{推论}
%   \newtheorem{lemma}[theorem]{引理}
%   \newtheorem{proposition}[theorem]{命题}
%   \theoremsymbol{\ensuremath{\mdlgwhtsquare}}
%   \newtheorem{theoremnoproof}[theorem]{定理}
%   \newtheorem{propositionnoproof}[theorem]{命题}
%   \theoremsymbol{}
%   \newtheorem{definition}[theorem]{定义}
%   \newtheorem{remark}[theorem]{注}
%   \newtheorem{example}[theorem]{例}
%   \theoremstyle{nonumberplain}
%   \theoremheaderfont{\bfseries}\theorembodyfont{\upshape}
%   \theoremindent0pt
%   \theoremsymbol{\ensuremath{\mdlgwhtsquare}}
%   \newtheorem{proof}{证明}
%   \theoremsymbol{}
%   % TODO: theorem style
% }

% 配置enumitem宏包
% \setlist*[enumerate,1]{left = \parindent .. 2\parindent} % 设置enumerate环境一级分项的缩进, 用star version以添加设置而不是覆盖之前的设置

% 浮动体控制宏包
% 确定浮动对象的位置, 可以使用 H, 强制将浮动对象放到这里 (很可能效果很差, 不推荐使用)
% 提供 \newfloat 命令建立新的浮动体环境 (与 captions 系列宏包相合性不好, 不推荐使用)
% \usepackage{float}
% 浮动图形控制宏包.
% 允许上一个 section 的浮动图形出现在下一个 section 的开始部分
% 该宏包提供处理浮动对象的 \FloatBarrier 命令,使所有未处
% 理的浮动图形立即被处理。这三个宏包仅供参考,未必使用:
\usepackage{placeins}
% \usepackage{floatflt} % 图文混排用宏包
% \usepackage{rotating} % 图形和表格的控制旋转

% 表格排版宏包
% 固定宽度的表格. 放在 hyperref 之前的话, tabularx 里的 footnote 显示不出来
% 也可以不使用额外的宏包而使用 {... @{\extracolsep{\fill}} ...} 对齐设置
% \usepackage{tabularx}
% 非浮动体表格与跨页表格
\usepackage{longtable}
% 从文件中读取并排版表格
\usepackage{pgfplotstable}% 文档类中设置了 \pgfplotsset{compat=1.18}

% 代码块排版宏包
% 推荐使用minted, 我们已经配置了针对各种情况进行了配置, 使用方法为:
% \begin{listing}
%   \caption{<caption>}\label{<label>}
%   \tcbinputminted{<language>}{<filename>}
% \end{listing}
% 如果想做进一步配置可以在 \tcbinputminted 的可选参数中传入 minted-options.
\usepackage{minted}

% 算法块排版宏包
% 我们针对 algorithmicx 包进行了配置, 对 algpseudocode 样式进行了测试
% 使用方法为:
% \begin{algorithm}[htbp]
% 	\caption{<caption>}\label{<label>}
% 	\begin{algorithmic}
%     <content>
% 	\end{algorithmic}
% \end{algorithm}
\usepackage{algpseudocode} % algorithmicx package with algpseudocode style


% %%%%%%%%%%%%%%%%%%%%%%%%%%%%%%%%%%%%%%%%%%%%%%%%%%%%%%%%%%%%%%%%%%%%%%%%%%% %
% 从这开始的代码是专门为了示例用的, 使用者可以全部注释掉
% %%%%%%%%%%%%%%%%%%%%%%%%%%%%%%%%%%%%%%%%%%%%%%%%%%%%%%%%%%%%%%%%%%%%%%%%%%% %
% 借用 ltxdoc 里面的几个命令方便写文档。
% \DeclareRobustCommand\cs[1]{\texttt{\char`\\#1}}
% \providecommand\pkg[1]{{\sffamily#1}}

% 借用 lipsum与zhlipsum 来填充内容
\usepackage{lipsum}
\usepackage{zhlipsum}

\XeTeXcharclass`┊=1

% 设置提到包与程序名字时候所用的字体
\newcommand{\pckg}[1]{\textsc{#1}}

% 表格示例的表题所用的宏
\newcommand{\unit}[1]{\mathrm{/#1}}
\newcommand{\pH}{\mathrm{pH}}
\newcommand{\hydrgen}{\mathrm{H}}
\newcommand{\hydrgenn}{\hydrgen^+}

% 在log里显示更多debug信息
\showboxdepth=3 \showboxbreadth=30
\overfullrule=1mm

% 表格中支持跨行
% \usepackage{multirow}

% 表格加脚注
% \usepackage{threeparttable}
% \pretocmd{\TPTnoteSettings}{\footnotesize}{}{}

% verbatim
\makeatletter
\newcommand{\verbatimfont}[1]{\renewcommand{\verbatim@font}{\ttfamily#1}}
\verbatimfont{\xiaowu}
\makeatother
\BeforeBeginEnvironment{verbatim}{\begin{tcolorbox}[breakable, nobeforeafter, before = {\vskip 9bp}, after = {\leavevmode\newline}]}
\AfterEndEnvironment{verbatim}{\end{tcolorbox}}

\AtBeginDocument{%https://tex.stackexchange.com/a/117995
  \DeclareMathOperator{\mspan}{span}
  \DeclareMathOperator{\Rq}{Rq}
  \DeclareMathOperator{\argmin}{argmin}
}
\newcommand*{\norm}[1]{\Vert #1 \Vert}

\AtBeginEnvironment{quote}{\itshape}
