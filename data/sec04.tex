% !TeX root = ../tjuthesis-example.tex

\section{杂项}

\subsection{正文字体测试}

参考例文中关于中英文字体的示例有点混乱. 中文摘要中以及正文的页脚中连标点符号和空格都是宋体的, 但是在参考文献中中文条目的文献标识号是英文字体的, 英文条目中括住文献标识号的方括号却是中文字体的, 甚至有些地方出现了等线字体, 可以说是非常不小心了.

{\bfseries 默认宋体的加粗设置为黑体}, 如果想要加粗的宋体文字的话, 请用\verb|\zhsong\bfseries|, {\zhsong\bfseries 例如\zhlipsum[1]}

{\bfseries\itshape 这是加粗斜体的字. These are bold italic letters. \zhlipsum[1]\lipsum[1]}

\subsection{数学字体测试}

这里有一些小写的希腊字母:
\begin{equation}
  \alpha \beta \gamma \delta \varepsilon \zeta \eta \theta \iota \kappa \lambda \mu \nu \xi o \pi \rho \sigma \tau \upsilon \varphi \chi \psi \omega
\end{equation}
这里有一些大写的希腊字母,他们应该都是斜体:
\begin{equation}
  A B \Gamma \Delta E Z H \Theta I K \Lambda M N \Xi O \Pi P \Sigma T \Upsilon \varPhi X \Psi \Omega
\end{equation}
这里有一些其它字母,其中偏导算子和梯度算子应该是正体:
\begin{equation}
  \aleph \partial \nabla
\end{equation}

% TODO: https://tex.stackexchange.com/questions/439921/using-a-different-font-for-digits-in-text-mode
除此之外, 数学模式中的数字和正文中的数字应该有相同的字体, 我们将数学模式中的数字也设置为了Times New Roman字体, 为此我们隐藏了fontspec宏包的警告. 效果如下:
\begin{equation}
  1234567890\text{1234567890}
\end{equation}
粗体版数字
\begin{equation}
  \symbf{1234567890}\text{\textbf{1234567890}}
\end{equation}

我们再来看一下其它字体族, 分别是cal, bfcal, scr和bfscr.
\begin{equation}
  \symcal{ABCDEFGHIJKLMNOPQRSTUVWXYZ}
\end{equation}
\begin{equation}
  \symbfcal{ABCDEFGHIJKLMNOPQRSTUVWXYZ}
\end{equation}
\begin{equation}
  \symscr{ABCDEFGHIJKLMNOPQRSTUVWXYZ}
\end{equation}
\begin{equation}
  \symbfscr{ABCDEFGHIJKLMNOPQRSTUVWXYZ}
\end{equation}

% TODO: setoperatorfont
% https://tex.stackexchange.com/questions/246304/changing-the-operator-font-with-unicode-math-loaded
% \begin{equation}
%   \sin \cos
% \end{equation}

\subsection{BibLaTeX 文献著录}\label{sec:exm-bib}

采用 \verb|\cite| 命令来引用文献, 我们已经将其修改为正确的引用格式. 对于数学专业, 引用文献不需要上标 \cite{atiyah_introduction_1969}, 外文文献的作者名字也不用按照国标示例全大写. 如果一本图书只引用同一处内容, 则可以在参考文献表中标注页码, 例如 \cite{herrlich_axiom_2006}; 如果图书广泛引用, 则不需要多次重复著录和标注页码, 即例如 \cite{atiyah_introduction_1969}. 为了示例的需要, 我们来多搞点文献, 使得一共有十个及以上的文献 \cite{jacobson_basic_1985, jacobson_basic_1989, zariski_commutative_1958, zariski_commutative_1960, ciarlet_linear_2013, flaherty_riemannian_1992, munkres_topology_2000, ahlfors_complex_1978, milne_algebraic_2017}\upcite{munkres_topology_2000}.

% \ExecuteBibliographyOptions

% 我们先引用一下郑博文写的书\cite{rudin1976principleschinese3}.

% 我们再引用一下张赫写的书\cite{rudin1976principleschinese}.

% 我们还可以两本书一起引用\cite{rudin1976principleschinese,rudin1976principleschinese3}.

% 最后我们看一下英文的参考文献\cite{rudin1976principles}.

% 多本书一起引用的效果是这样的\cite{rudin1976principleschinese,rudin1976principleschinese3, rudin1976principles, rudin1976principleschinese2}.

测试一下翻译成外文的外文文献的著录 \cite{sally_history_1985}. 不太正式的参考资料\footfullcite{andrew_how_2016}我们在脚注里引用, 并且不著录进参考文献表中. 目前这个引用的著录格式与参考文献表中要求的格式不太相同, 不过短时间没有解决这个问题的动力.

\zhlipsum[1]

\subsection{脚注测试}

这里有一段文字\footnote{这是脚注1.}, 这里还有一段文字\footnote{这是脚注2.}. 脚注格式应该是由不上标的圆圈圈住的数字再加上内容.

\zhlipsum

\subsection{分项测试}

\begin{enumerate}
  \item 一级分项要求用左右括号与数字的形式.
  \begin{enumerate}
    \item 二次分项要求用数字与右括号的形式.
  \end{enumerate}
\end{enumerate}

\zhlipsum[1]
