% !TeX root = ../tjuthesis-example.tex

% 如果想要仿照目录在 "绪论" 二字之间添加空格,
%   应使用\hspace*{0.5\ccwd}插入半个汉字宽度的空白
\section{绪论}
% \section{绪\hspace*{0.5\ccwd}论}

\subsection{概述}

本文是同济大学学位论文 LaTeX 模板 {\version} 版本的使用示例文档. 模板创作受到 \href{https://github.com/tuna/thuthesis}{tuna/thuthesis} 启发, 可以大体满足 2021 届数学与应用数学专业及其它理工类专业的毕业设计格式要求, 并提供了一些接口以便用户进行对文档类定制样式的配置以及新内容的排版. 硕博的论文模板目前可以先参见 \href{https://github.com/marquistj13/TongjiThesis}{marquistj13/TongjiThesis}.

模板创作的主要起源在于数学与应用数学专业下发了一个 tex 文件并要求所有学生以该文件为例进行毕业设计的撰写. 那个文件中样式与内容耦合得比较厉害, 与学校的格式要求出入也比较大, 同时在网络上也没有找到一份出色的模板, 于是就萌生出了制作这份模板的念头. 这一学期 (2020--2021 学年春季学期) 中我也是将创作这份模板当作自己的第二份毕业设计来做, 付出了很多努力. 希望这份模板可以为数学与应用数学专业的学生以及其它希望用 LaTeX 排版毕业设计的同学提供帮助.

本文是使用示例文档, 而非模板的文档, 重在叙述格式要求以及给出排版示例. 关于模板里具体如何实现, 引用了哪些宏包, 对宏包做了哪些配置, 如果用户想要知道的话可以查看源码或向作者提问. 本文的第二章介绍了手册对格式的要求, 第三章对一些论文中经常出现的内容给出了排版示例, 第四章是一些尚未完成的内容, 第五章进行了总结并展望了模板未来的发展.

\subsection{模板使用与更新}

目前模板只能通过本地免安装的方法使用, 使用方法为从 GitHub 下载最新版, 原则上只需要将 tjuthesis.cls 放入工作目录中即可使用, 用户也可以将包含示例的整个文件夹作为工作目录, 通过修改示例来使用.

更新时读者只需要重新下载最新版的 tjuthesis.cls 文件并替换旧的文件, 如果大版本号有更新的话可能其它 tex 文件也有更改的必要, 届时我会在更新中说明应如何更改.

本模板的编译需要 2020-10-01 及以后版本的 LaTeX 内核, 并使用 XeLaTeX 引擎, 建议用户安装 2021 及以后版本的 TeXLive 发行版. 建议使用 LaTeXmk 进行编译, 需要安装好 \href{https://www.perl.org/get.html}{Perl}. 如果选用 \pckg{minted} 宏包排版代码块, 则需要安装好 \href{https://wiki.python.org/moin/BeginnersGuide/Download}{Python} 以及 Python 中的 \href{https://pygments.org/download/}{Pygments} 包, 并在编译时添加 --shell-escape 选项. 完整的编译命令示例如下:
\begin{verbatim}
  latexmk -synctex=1 -interaction=nonstopmode -file-line-error --shell-escape -xelatex <filename>
\end{verbatim}
