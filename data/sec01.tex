% !TeX root = ../tjuthesis-example.tex

\section{字体测试}

\zhlipsum[1]

\subsection{正文字体测试}

{\bfseries 默认宋体的加粗设置为黑体}, 如果想要加粗的宋体文字的话, 请用\verb|\zhsong\bfseries|, {\zhsong\bfseries 例如\zhlipsum[1]}

{\bfseries\itshape 这是加粗斜体的字. These are bold italic letters. \zhlipsum[1]\lipsum[1]}

\subsection{数学字体测试}

这里有一些小写的希腊字母:
\begin{equation}
  \alpha \beta \gamma \delta \varepsilon \zeta \eta \theta \iota \kappa \lambda \mu \nu \xi o \pi \rho \sigma \tau \upsilon \varphi \chi \psi \omega
\end{equation}
这里有一些大写的希腊字母,他们应该都是斜体:
\begin{equation}
  A B \Gamma \Delta E Z H \Theta I K \Lambda M N \Xi O \Pi P \Sigma T \Upsilon \varPhi X \Psi \Omega
\end{equation}
这里有一些其它字母,其中偏导算子和梯度算子应该是正体:
\begin{equation}
  \aleph \partial \nabla
\end{equation}

除此之外, 数学模式中的数字和正文中的数字应该有相同的字体, 我们将数学模式中的数字也设置为了Computer Modern字体, 为此我们隐藏了fontspec宏包的警告. 效果如下:
\begin{equation}
  1234567890\text{1234567890}
\end{equation}
粗体版数字
\begin{equation}
  \symbf{1234567890}\text{\textbf{1234567890}}
\end{equation}

我们再来看一下其它字体族, 分别是cal, bfcal, scr和bfscr.
\begin{equation}
  \symcal{ABCDEFGHIJKLMNOPQRSTUVWXYZ}
\end{equation}
\begin{equation}
  \symbfcal{ABCDEFGHIJKLMNOPQRSTUVWXYZ}
\end{equation}
\begin{equation}
  \symscr{ABCDEFGHIJKLMNOPQRSTUVWXYZ}
\end{equation}
\begin{equation}
  \symbfscr{ABCDEFGHIJKLMNOPQRSTUVWXYZ}
\end{equation}

\subsection{第一节第三子节的标题}

\zhlipsum
