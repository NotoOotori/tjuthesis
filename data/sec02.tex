% !TeX root = ../tjuthesis-example.tex

\section{文档格式}

\subsection{页面格式}

\subsubsection{装订线}

装订线由 ``装订线" 三字与unicode字符 U+250A ``┊" 构成, 其中汉字两两之间间隔五个 ``┊" 字符, 而汉字外侧还有十三个 ``┊" 字符. 经过测量, 我们发现装订线整体距离页上边框约为5.89厘米, 距离页下边框约5.02厘米, 而装订线水平中心距离页左边框约1.88厘米, 装订线的行间距约为0.48厘米. 根据 \pckg{fancybox} 的文档, 我们知道参数中默认页面上有1英寸的边缘, 经过测量我们发现页面左边的边缘约为2.74厘米, 于是我们可以采用 \verb|\fancyput(-0.86cm,-12.745cm){<gutter-name>}| 的方法来绘制装订线. 不过用宏 \verb|\fancyput| 来绘制装订线会与verbatim环境有冲突, 比如我们已经观察到多页的代码会导致装订线有奇怪的表现, 虽然可以通过一些技巧来解决, 但是我们仍有些不满意.

\subsection{几何与页眉页脚}

经过测量, 我们发现页眉线距离页上边框约为1.33英寸, 距离页左边框约为1.36英寸, 距离页右边框约为0.74英寸, 页脚线距离页下边框约为0.80英寸, 于是经过计算与调整, 我们得到了在 \pckg{geometry} 宏包中应提供的几何数据. 这里我们偷懒了, 没有按照推荐选用 \verb|includeall| 选项, 在今后有必要的话我们可能会进行修复. 此外, 由于我们需要在页眉中放置同济大学的图标, 因此设置了 \verb|headheight = 45pt|.

页眉页脚我们采用 \pckg{fancyhdr} 进行配置, 页眉页脚中出现的文字均为小四大小, 并且为宋体字体. 关于页眉, 注意到 ``毕业设计(论文)" 要比页眉线高出一些, 经测量字的顶端距离页眉线约有0.38英寸, 因此我们需要将这几个字用 \verb|\raisebox| 宏抬升0.15英寸左右. 关于页脚, 要注意摘要和目录页是用大写罗马数字编号, 并且需要采用 \verb|\raisebox| 下沉0.02英寸左右, 而正文用阿拉伯数字从一开始重新编号, 并且 ``共 页" 与 ``第 页" 之间的间距要大于一个汉字字符宽度, 因此经过测量我们采用 \verb|\hspace*{0.21in}| 来添加这段空白. 页脚中的阿拉伯数字我们还没能做成宋体字体, 这个只能先搁置了.

\subsection{中文摘要页}

摘要页包含标题, 摘要以及关键词. 我们先来整理一下手册的要求. 手册中第十七页中提到:
\begin{quote}
  课题名称应该简明, 突出主题. 如字数太多, 可分列成主标题和副标题. 字体, 字号详见附件.

  论文内容摘要主要是对撰写过程中实践, 实验, 研究的内容, 方法和得到的主要结果的完整概括, 中文字数一般为300字左右, 并应有相应的英译文.

  设计总说明主要阐述本设计的基础条件, 技术要求, 基本数据, 效果分析 (经济, 社会, 人文等方面) 及简要结论, 中文字数一般为1500字左右及300字左右的英文摘要. 个别无设计说明书的专业, 也应有300字左右的英文设计简介.

  关键词一般3--5个为宜. 字体, 字号参见附件.
\end{quote}
手册中第四十四页的打印格式及要求说明中提到:
\begin{quote}
  标题栏居中书写, 黑体, 小二号加粗 (副标题为三号).
\end{quote}
手册中第四十五页的参考例文中对中文摘要页有如下要求:
\begin{quote}
  课题名称: 小二号, 黑体, 加粗, 居中, 行距18磅, 段前0.5行, 段后0.5行. 上下各空一行.
  摘要二字: 四号, 黑体, 居中. 行距18磅. 段前0.5行, 段后0.5行.
  摘要正文300字左右, 五号宋体, 首行缩进2个汉字符. 行距18磅.
  摘要与关键词之间空一行.
  关键词三字及冒号: 五号宋体, 加粗.
  关键词3--5个, 五号宋体. 逗号分开, 最后一个关键词后面无标点符号.
\end{quote}

在摘要页的排版中, 我们遇到了一系列 Word 中的术语. 经过研究, 我们发现 Word 中所谓一 ``行" 的高度约为小五号字体大小 (9bp) 的1.2倍, 需要注意的是 ``空一行" 之后我们仍需要添加行距所带来的竖直空白. ``摘要" 二字间距约为0.10英寸, 我们暂时没有找到合适的空格, 就直接用 \verb|\hspace*{0.10in}| 来添加这段水平空白. 关键词中我们尊重样例而采用中文标点符号. 标题和 ``关键词" 我们需要用到 \pckg{fontspec} 宏包中的 \verb|FakeBold| 特性, 我们设置了 \verb|EmboldenFactor = 4|, 这与示例的效果相类似.

\subsection{章节标题测试}

示例中说一级标题应换页. 17级应数专业的示例中, 一级标题为小三加粗 (中文字为黑体不加粗, 下同), 二级标题与三级标题均为五号加粗, 序号与标题之间的空格偏大, 应该为两个中文字符 (即 \verb|2\ccwd| ) 的长度.

\subsubsection{第三级标题}

第三级标题应空两格再写序号.

\zhlipsum[1]

\subsection{BibLaTeX 文献著录}

采用 \verb|\cite| 命令来引用文献, 我们已经将其修改为正确的引用格式. 对于数学专业, 引用文献不需要上标 \cite{atiyah_introduction_1969}, 外文文献的作者名字也不用按照国标示例全大写. 如果一本图书只引用同一处内容, 则可以在参考文献表中标注页码, 例如 \cite{herrlich_axiom_2006}; 如果图书广泛引用, 则不需要多次重复著录和标注页码, 即例如 \cite{atiyah_introduction_1969}.

% 我们先引用一下郑博文写的书\cite{rudin1976principleschinese3}.

% 我们再引用一下张赫写的书\cite{rudin1976principleschinese}.

% 我们还可以两本书一起引用\cite{rudin1976principleschinese,rudin1976principleschinese3}.

% 最后我们看一下英文的参考文献\cite{rudin1976principles}.

% 多本书一起引用的效果是这样的\cite{rudin1976principleschinese,rudin1976principleschinese3, rudin1976principles, rudin1976principleschinese2}.

测试一下翻译成外文的外文文献的著录 \cite{sally_history_1985}. 不太正式的参考资料\footfullcite{andrew_how_2016}我们在脚注里引用, 并且不著录进参考文献表中. 目前这个引用的著录格式与参考文献表中要求的格式不太相同, 不过短时间没有解决这个问题的动力.

\zhlipsum[1]

\subsection{脚注测试}

这里有一段文字\footnote{这是脚注1.}, 这里还有一段文字\footnote{这是脚注2.}. 脚注格式应该是由不上标的圆圈圈住的数字再加上内容.

\zhlipsum

\subsection{分项测试}

\begin{enumerate}
  \item 一级分项要求用左右括号与数字的形式.
  \begin{enumerate}
    \item 二次分项要求用数字与右括号的形式.
  \end{enumerate}
\end{enumerate}

\zhlipsum[1]
