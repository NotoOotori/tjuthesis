% !TeX root = ../tjuthesis-example.tex

\section{文档格式}

\subsection{页面格式}

\subsubsection{装订线}

装订线由``装订线"三字与unicode字符U+250A``┊"构成, 其中汉字两两之间间隔五个``┊"字符, 而汉字外侧还有十三个``┊"字符. 经过测量, 我们发现装订线整体距离页上边框约为5.89厘米, 距离页下边框约5.02厘米, 而装订线水平中心距离页左边框约1.88厘米, 装订线的行间距约为0.48厘米. 根据\pckg{fancybox}的文档, 我们知道参数中默认页面上有1英寸的边缘, 经过测量我们发现页面左边的边缘约为2.74厘米, 于是我们可以采用 \verb|\fancyput(-0.86cm,-12.745cm){<gutter-name>}| 的方法来绘制装订线. 不过用宏 \verb|\fancyput| 来绘制装订线会与verbatim环境有冲突, 比如我们已经观察到多页的代码会导致装订线有奇怪的表现, 虽然可以通过一些技巧来解决, 但是我们仍有些不满意.

\subsection{章节标题测试}

示例中说一级标题应换页. 17级应数专业的示例中, 一级标题为小三加粗 (中文字为黑体不加粗, 下同), 二级标题与三级标题均为五号加粗, 序号与标题之间的空格偏大, 应该为两个中文字符 (即 \verb|2\ccwd| ) 的长度.

\subsubsection{第三级标题}

第三级标题应空两格再写序号.

\zhlipsum[1]

\subsection{BibLaTeX 文献著录}

采用 \verb|\cite| 命令来引用文献, 我们已经将其修改为正确的引用格式. 对于数学专业, 引用文献不需要上标 \cite{atiyah_introduction_1969}, 外文文献的作者名字也不用按照国标示例全大写. 如果一本图书只引用同一处内容, 则可以在参考文献表中标注页码, 例如 \cite{herrlich_axiom_2006}; 如果图书广泛引用, 则不需要多次重复著录和标注页码, 即例如 \cite{atiyah_introduction_1969}.

% 我们先引用一下郑博文写的书\cite{rudin1976principleschinese3}.

% 我们再引用一下张赫写的书\cite{rudin1976principleschinese}.

% 我们还可以两本书一起引用\cite{rudin1976principleschinese,rudin1976principleschinese3}.

% 最后我们看一下英文的参考文献\cite{rudin1976principles}.

% 多本书一起引用的效果是这样的\cite{rudin1976principleschinese,rudin1976principleschinese3, rudin1976principles, rudin1976principleschinese2}.

测试一下翻译成外文的外文文献的著录 \cite{sally_history_1985}. 不太正式的参考资料\footfullcite{andrew_how_2016}我们在脚注里引用, 并且不著录进参考文献表中. 目前这个引用的著录格式与参考文献表中要求的格式不太相同, 不过短时间没有解决这个问题的动力.

\zhlipsum[1]

\subsection{脚注测试}

这里有一段文字\footnote{这是脚注1.}, 这里还有一段文字\footnote{这是脚注2.}. 脚注格式应该是由不上标的圆圈圈住的数字再加上内容.

\zhlipsum

\subsection{分项测试}

\begin{enumerate}
  \item 一级分项要求用左右括号与数字的形式.
  \begin{enumerate}
    \item 二次分项要求用数字与右括号的形式.
  \end{enumerate}
\end{enumerate}

\zhlipsum[1]
