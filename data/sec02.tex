% !TeX root = ../tjuthesis-example.tex

\section{文档格式}

\subsection{页面格式}

\subsubsection{装订线}

装订线由 ``装订线" 三字与unicode字符 U+250A ``┊" 构成, 其中汉字两两之间间隔五个 ``┊" 字符, 而汉字外侧还有十三个 ``┊" 字符. 经过测量, 我们发现装订线整体距离页上边框约为5.89厘米, 距离页下边框约5.02厘米, 而装订线水平中心距离页左边框约1.88厘米, 装订线的行间距约为0.48厘米. 根据 \pckg{fancybox} 的文档, 我们知道参数中默认页面上有1英寸的边缘, 经过测量我们发现页面左边的边缘约为2.74厘米, 于是我们可以采用 \verb|\fancyput(-0.86cm,-12.745cm){<gutter-name>}| 的方法来绘制装订线. 不过用宏 \verb|\fancyput| 来绘制装订线会与verbatim环境有冲突, 比如我们已经观察到多页的代码会导致装订线有奇怪的表现, 虽然可以通过一些技巧来解决, 但是我们仍有些不满意.

\subsubsection{几何与页眉页脚}

经过测量, 我们发现页眉线距离页上边框约为1.33英寸, 距离页左边框约为1.36英寸, 距离页右边框约为0.74英寸, 页脚线距离页下边框约为0.80英寸, 于是经过计算与调整, 我们得到了在 \pckg{geometry} 宏包中应提供的几何数据. 这里我们偷懒了, 没有按照推荐选用 \verb|includeall| 选项, 在今后有必要的话我们可能会进行修复. 此外, 由于我们需要在页眉中放置同济大学的图标, 因此设置了 \verb|headheight = 45pt|.

页眉页脚我们采用 \pckg{fancyhdr} 进行配置, 页眉页脚中出现的文字及数字均为小四大小, 并且都是宋体字体. 关于页眉, 注意到 ``毕业设计(论文)" 要比页眉线高出一些, 经测量字的顶端距离页眉线约有0.38英寸, 因此我们需要将这几个字用 \verb|\raisebox| 宏抬升0.15英寸左右. 关于页脚, 要注意摘要和目录页是用大写罗马数字编号, 并且需要采用 \verb|\raisebox| 下沉0.02英寸左右, 而正文用阿拉伯数字从一开始重新编号, 并且 ``共 页" 与 ``第 页" 内部文本与数字的间距为一个汉字宽度, 而它们之间的间距为一点五个汉字宽度.

\subsection{摘要页}

摘要页包含标题, 摘要以及关键词. 我们先来整理一下手册的要求. 手册中第十七页中提到:
\begin{quote}
  课题名称应该简明, 突出主题. 如字数太多, 可分列成主标题和副标题. 字体, 字号详见附件.

  论文内容摘要主要是对撰写过程中实践, 实验, 研究的内容, 方法和得到的主要结果的完整概括, 中文字数一般为300字左右, 并应有相应的英译文.

  设计总说明主要阐述本设计的基础条件, 技术要求, 基本数据, 效果分析 (经济, 社会, 人文等方面) 及简要结论, 中文字数一般为1500字左右及300字左右的英文摘要. 个别无设计说明书的专业, 也应有300字左右的英文设计简介.

  关键词一般3--5个为宜. 字体, 字号参见附件.
\end{quote}
手册中第四十四页的打印格式及要求说明中提到:
\begin{quote}
  标题栏居中书写, 黑体, 小二号加粗 (副标题为三号).
\end{quote}
手册中第四十五页的参考例文中对中文摘要页有如下要求:
\begin{quote}
  课题名称: 小二号, 黑体, 加粗, 居中, 行距18磅, 段前0.5行, 段后0.5行. 上下各空一行.
  ``摘要" 二字: 四号, 黑体, 居中. 行距18磅. 段前0.5行, 段后0.5行.
  摘要正文300字左右, 五号宋体, 首行缩进2个汉字符. 行距18磅.
  摘要与关键词之间空一行.
  ``关键词" 三字及冒号: 五号宋体, 加粗.
  关键词3--5个, 五号宋体. 逗号分开, 最后一个关键词后面无标点符号.
\end{quote}
手册中第四十六页的参考例文中对英文摘要页有如下要求:
\begin{quote}
  英文课题名称: 换页. 小二号, Times New Roman, 加粗, 居中, 行距18磅, 段前0.5行, 段后0.5行, 上下各空一行.
  ``ABSTRACT" 一词: 四号 Times New Roman 居中, 段前0.5行, 段后0.5行. 行距18磅.
  摘要正文: 五号 Times New Roman, 首行缩进2个汉字符, 行距18磅.
  摘要与关键词之间空一行.
  ``Key words" 两词及冒号: 五号 Times New Roman, 加粗.
  关键词: 五号 Times New Roman, 各关键词之间逗号分开, 逗号后加一空格. 行距18磅.
\end{quote}

在摘要页的排版中, 我们遇到了一系列 Word 中的术语. 经过研究, 我们发现 Word 中所谓一 ``行" 的高度约为小五号字体大小 (9bp) 的1.2倍, 需要注意的是 ``空一行" 之后我们仍需要添加行距所带来的竖直空白.

中文摘要页中, ``摘要" 二字间距为半个汉字宽度, 因此我们用 \verb|\hspace*{0.5\ccwd}| 来添加这段水平空白. 关键词中我们尊重例文而采用中文标点符号. 标题和 ``关键词" 我们需要用到 \pckg{fontspec} 宏包中的 \verb|FakeBold| 特性, 我们设置了 \verb|EmboldenFactor = 4|, 这与例文的效果相类似. \emph{似乎中文摘要页中的空格之类的本应该用西文字体的符号也都采用了中文字体, 我们并未模仿这一行为.}

英文摘要页中, 参考例文中有两处排版问题. 一是 ``ABSTRACT" 一词例文的注记中未要求加粗, 但是例文中加粗了; 二是 ``Key words" 一词后的冒号例文中采用了宋体的全角冒号, 这似乎不合适. \emph{因此我们改正了这两处排版问题, 使 ``ABSTRACT" 变为不加粗, 而将 ``Key words" 后的中文全角冒号改为英文的半角冒号及一个空格.}

\subsection{章节标题测试}

示例中说一级标题应换页. 17级应数专业的示例中, 一级标题为小三加粗 (中文字为黑体不加粗, 下同), 二级标题与三级标题均为五号加粗, 序号与标题之间的空格偏大, 应该为两个中文字符 (即 \verb|2\ccwd| ) 的长度.

\subsubsection{第三级标题}

第三级标题应空两格再写序号.

\zhlipsum[1]
